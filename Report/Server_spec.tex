%%This is a very basic article template.
%%There is just one section and two subsections.
\documentclass[9pt,twocolumn]{article}

\begin{document}

\section{Server Semantics}

\subsection{Server Overview}

In order to negotiate the connection setup between two or more clients, we decided to
design and implement a server for clients to connect to. The server is meant to
facilitate the clients' connection status, similar to an instant messaging server. It
provides a method for clients to sign on or off, a way to check friends' online
statuses, and a method to request a friend's ip address and port.

\subsection{Design Decisions}

We had to make a number of design decisions for the server, which are described here.\\

We'll start with the choice of language. Because the server is a separate entity from
the chat client, we had some extra freedom for language and design choice. After all,
the only real requirement for the server was that it be able to connect and communicate
with each client. After some consideration, we decided to implement the server in
Python for two main benefits. First, Python has a number of modules, including a 
socket-based asynchronous server, available to aid with development. Second, Python's
general syntax and ease of use makes it a great tool for creating mock-ups or
relatively small, short-term project. Of course, it works well for larger projects too.
Python made coding much cleaner and more maintainable for the short term goals of our
project.\\

We wanted the server to perform some of the same capabilities that a watered-down chat
server (such as one used for Skype) might have or use. To that end, we needed it to
have functionality such as user login and tracking. We determined that the server
had to perform the following tasks at minimum:
\begin{enumerate}
  \item Provide a way for a user to register that he is online.
  \item Provide a way for users to talk to each other.
  \item Keep track of who the user is allowed to talk to.
  \item Provide information updates to each user about who he can talk to.
\end{enumerate}

Lets explain each of these points. The first point is essential because it represents
our intention to add a server element in the first place: to keep track of users.
By giving users a way to register that they are connected (and conversely,
disconnected as they become so), the server is given the ability to keep track of
users as they log in or log out, allowing it to perform additional functionality.\\

The second point is important because, in a typical setting, users aren't going to
know the IP address or port number of other machines they want to communicate with.
The client needs a way of obtaining this information from the server, and thus being
able to chat with others.\\

Strictly speaking, only points 1 and 2 above are actually needed for a minimalistic
server to function. The server keeps track of who is online, and tells clients where
they can contact other clients who are online. However, we wanted our server to
emulate a couple more features of industry programs. This leads to our third point,
which we can pretty much sum up as adding a friend list into the server's duties.
The server would keep a list of who is friends with who, thereby limiting clients
from being able to randomly spam other connected users. \\

The final piece of core functionality is closely related to point 3 above. In order
to cache friend information on the client side (so the client won't make requests
for offline users all the time), the server should be able to send updates to each
connected client to let them know who they are able to contact at any given time. \\ 

Our final core design choice involved storage of user information. As it is rather standard
in industry, we stored information in a separate database file. In this database, the server
would keep track of user and friend list information, as well as any other necessary
information that we needed. Although it wasn't very much information to keep track of, 
it still gave us a relatively simple method of keeping track of users without having
 to manually parse text or config files.\\
 
\subsection{Server Implementation}
Now that we've covered the key design decisions of the server, let's describe how the server
was actually implemented.\\

Python has a number of different modules available for an application to communicate
through network sockets. Our first consideration was the SocketServer module. This library
provided an interface which would block to receive requests, and then execute a request
handler via a callback. Unfortunately this module had one major flaw: it could only handle
one user request at a time. Because the server needed to handle multiple connected users
at once, we eventually settled on using the asyncore module. This module has essentially
the same functionality as SocketServer, but with asynchronous I/O. In other words, each
new incoming request would execute a request handler in its own thread, allowing multiple
clients to be connected and logged in at once.\\

In order to allow clients to log on, request information, and receive updates, it made
sense for the server to use a connection oriented protocol. Thus, TCP was the natural
choice. When a client connects to the server, the server spawns a new thread which maintains
the connection to the client. If the client makes any more requests, packets will be sent
directly to this child thread instead of the main server loop. Each child thread has its own
instance of a request handler, which handles the logic for reading, interpreting, and replying
to each client request. If a client closes its connection, the thread closes as well and is
removed from the loop. \\

To log in, a client establishes a connection to the server (in our tests
the client would accept the IP address of the server as an input parameter) and sends it
a packet with two 20 byte strings, a username and a password, padded as necessary with
null bytes. The server then accepts the packet, verifies the username and password, and
replies back to the client with either a success or failure packet depending on the
validity of the credentials. The password is sent over the network as plaintext (normally
it would be encapsulated in an SSL connection) and the server, upon reception, hashes
the password and a salt with an SHA-512 function and compares the result to a value in
the database. If the hashes match, the user would be successfully logged in. This
security isn't very hack-proof, but it at least served to help us consider more
about network security in general.\\

Once a client successfully logs in, the server replies back with a friend list packet.
This packet contains a list of usernames and ids of all the users a client is currently
friends with. This packet does not contain any online status, and in the event of
overflow (more friends than can fit in the packet) the server would simply truncate
the list. We did not run into any issues with this in testing because we only had 2
users, but in a real-world environment this certainly would be more of a concern.\\

Every few seconds (our implementation configured it to 4), the server sends out an
update to each client currently maintaining a connection to it. To accomplish this,
the server initializes a timer to provide an interrupt and execute a special method
once per interval. This method, when run, determines who has changed status in the past
interval (whether they went online or offline), and sends each of the clients who is
friends with that person a status update. To reduce the overhead of using the same
connection, this status update would be sent via a separate connection that each client
establishes when it first connects. When the method finishes executing, it uses a callback
to set up a new timer interrupt. \\

One other capability the server provides to each client is a friend address request. Due to
the nature of our project and the desire to not want to endlessly enter IP/port information
when performing tests, we implemented the ability for a client to request a friend's IP/port
information from the server. A client simply sends a packet containing a friend's ID and the
server, upon verifying the friend is indeed a friend and is online, replies with the
IP address and port number in which friend is listening for connections. \\

For sending and receiving binary data over the network, the server uses the Python struct
module, which provides a convenient way to pack and unpack data with specific size
limits. When sending data, the server calculates the values it needs, packs them as
appropriate, and concatenates the values together (the struct module uses strings
for its binary representation, in which encoded character represent the specific values
we want to send or receive). When receiving data, the server does this in reverse. \\

To keep track of users and friends, the server uses a SQLite database consisting of
two tables. The first table, called ``Users,'' has one row per user, with columns
giving the username, user ID, and password hash. The second table, called ``Friends,''
contains one row for every friend a user has, for every user. The design of this is such
that ``A is friends with B'' and ``B is friends with A'' needs to occur in the table
for A and B to properly communicate. In other words, for every set of friends, two rows
in the ``Friends'' table are needed. This is not exactly the most efficient (or even the
most reliable) way to keep track of the information, but it served its purpose well enough
for the scope of this project. When the server wants to obtain information from the
database, it uses the sqlite3 Python module to do so. \\

\end{document}
