%%%%%%%%%%%%%%%%%%%% The_Report.tex %%%%%%%%%%%%%%%%%%%%%%%%%%%%%%%%%%%
%
% The AudioFiles Voice Chatting Report
%
% author: William McVicker, Tim Biggs, Chris Hoover
%
%%%%%%%%%%%%%%%% Springer %%%%%%%%%%%%%%%%%%%%%%%%%%%%%%%%%%


% RECOMMENDED %%%%%%%%%%%%%%%%%%%%%%%%%%%%%%%%%%%%%%%%%%%%%%%%%%%
\documentclass[letterpaper, 9 pt, balance, conference]{ieeeconf} 

\usepackage{float}
\usepackage{amsmath}        % need for equations
\usepackage{mathptmx}       % selects Times Roman as basic font
\usepackage{helvet}         % selects Helvetica as sans-serif font
\usepackage{courier}        % selects Courier as typewriter font
%
\usepackage{makeidx}         % allows index generation
\usepackage{graphicx}        % standard LaTeX graphics tool
\usepackage{subfig}
                             % when including figure files
\usepackage{multicol}        % used for the two-column index
\usepackage{balance}
% see the list of further useful packages
% in the Reference Guide

\makeindex             % used for the subject index
                       % please use the style svind.ist with
                       % your makeindex program

\newcommand{\thickhline}{\noalign{\hrule height 1.0pt}}
\newcommand{\tab}{$\hspace{6pt}$}
\newcommand{\mathbi}[1]{\textbf{\em #1}}
%%%%%%%%%%%%%%%%%%%%%%%%%%%%%%%%%%%%%%%%%%%%%%%%%%%%%%%%%%%%%%%%%%%%%%%%%%%%%%%%%%%%%%%%%

\begin{document}

\bibliographystyle{IEEEtran}

\title{Audio Data Transmission using DCCP Transport Protocol}

\author{\authorblockN{Chris Hoover\authorrefmark{1}, Tim Biggs\authorrefmark{1}, William McVicker\authorrefmark{1}}$\vspace{3pt}$
\authorblockA{\authorrefmark{1}California Polytechnic State University\\
   San Luis Obispo, CA 93407 USA\\%
   Email: chhover@, tebiggs@, wmcvicke@calpoly.edu}%
}

\maketitle
\IEEEpeerreviewmaketitle

\begin{abstract}
\boldmath 
This paper aims to build a framework that utilizes the Data Congestion Control Protocol (DCCP) in order to experimentally test its congestion control system.  DCCP is a hybrid between TCP and UDP in that it offers congestion control across an unreliable transmission protocol along with a connection-oriented setup and teardown design. Applications that do not require a reliable connection, but desire to trasmit data at high rates are the ideal users of DCCP, i.e. audio streaming, video streaming, and video games. This paper proposes the design of an audio chatting application that has a framework capable of swapping out different transport layer protocols.  Specifically, three protocols were selected: DCCP, TCP, and UDP.  By abstracting out the transport layer at the application level, we can easily compare and contrast the differences in jitter and packet loss between the three different transport layer protocols in an audio streaming/chatting environment.

\end{abstract}

\section{Introduction}
\label{sec:intro}

Data Congestion Control Protocol (DCCP) is a transport layer protocol that
implements congestion control over an unreliable network~\cite{kohler06}. 
This protocol is a hybrid between UDP and TCP in that it contains the benefits
of unreliability provided by UDP along with the congestion control system found in 
TCP. DCCP originated due to the problem that occurs when transmitting data
at high rates across a network with overloading background traffic. In this
scenario, UDP will continue to transmit packets over the network, which will 
likely be dropped and increase network congestion while TCP would increasingly
slow down and continue to try and resend packets that may not be relevant by 
the time they arrive at their destination such as in meadia streaming 
applications.

DCCP is currently at the proposed standard RFC status (4340-4342).  It is 
currently available for use through the linux socket library~\cite{dccp_website}.  
Its main goal was "to give streaming UDP applications little reason
not to switch to DCCP"~\cite{dccp_wg} by requiring as little overhead as 
possible with full functionality.  The design of DCCP was geared mostly 
toward media streaming applications which include audio, video, and gaming
software where responsiveness is traded for a steadier, less bursty network
connection.

The main goal of this paper is to evaluate DCCP's congestion control performance
by implementing a voice chatting application with DCCP as the underlying audio
transmission protocol.  The voice chatting framework was designed to allow the
user, upon running the application, choose between DCCP, UDP, and TCP as the
transport layer protocol of choice for transmitting audio packets. This is done
by building a framework that abstracts out the transport layer protocol at the
application layer. With this framework, we can observe differences in speed, 
jitter, and packet loss.


\section{Background}
\label{sec:backg}

The most popular transport layer protocols in use are TCP and UDP.  Both of these
protocols have their places in the networks realm and perform well under their 
respective applications.  TCP is known mainly for it reliability along with its
congestion control system while UDP favors timeliness over reliability.  With the
increase in media streaming over the past decade, the idea of timeliness has
become a very important feature.  We as humans, would rather compromise video
quality for synchronized audio and video mainly due to our abilities to deduce
information from segments of information.  Therefore, using a protocol such as UDP
that doesn't worry about loosing a few packets over time is preferred for this type 
of application.  However, when transmitting data that would be corrupt if pieces 
were lost, a reliable protocol must be used to maintain file integrity.  TCP was
designed for this very reason and is widely used as the standard reliable transport
layer protocol.

Even though both of these protocols cover the two extremes: reliable and unreliable
networks, at times a hybrid of the two is desireable.  A situation
that may arise is when media streaming occurs on a congested network. 
Congested networks cause extreme amounts of delay to TCP connections because of
the increased packet loss and the recovery TCP takes to maintain its reliability. 
So using TCP automatically is removed from the desireable options.  For
UDP connections, we can see that there is no traffic speed limit to restrict the
transmission rate UDP utilizes aside from the physical restrictions.  Therefore, 
applications that use UDP actually
may cause a network to become congested due to the high bursts that occur when
trasmitting large blocks of data.  Streaming video with encoding is a good example 
of applications that cause a varying datagram size, specifically MPEG's key frames
verus incremental frames.

DCCP was designed for this very purpose: to support congestion control on an 
unreliable network.  Not only do the authors of DCCP succesfully implement an 
unreliable transport layer protocol with congestion control support, but it also 
sheds light on the complexity of TCP.

\section{Architecture}
\label{sec:architec}

\begin{figure}[h]
   \centering
      \includegraphics[width=0.4\textwidth]{pics/setup}
   \caption{High-level diagram of how the voice chatting application is setup.}
\label{fig:setup}
\end{figure}

The voice chatting application was designed to work with many different clients
that communicate with a central server to retrieve information about how to call
other clients as well as the current status of the user's friends. The server 
maintains a database used to authenticate each user upon starting the application.
After logging into the system, the server is used as an intermediary for the clients
to retrieve current information about the user's friends' status, hostname, and
port number. This information can be used to request a chatting session with another
client that is logged onto the server.  Fig.~\ref{fig:setup} shows a high-level
network topology representation of what a set of clients logged into this chatting 
application would look like.  The server and client subsystems are described in
the following sections.


\section{Server Overview}

In order to negotiate the connection setup between two or more clients, we decided to
design and implement a server for clients to connect to. The server 
facilitates the clients' connections statuses, similar to an instant messaging server. It
provides a method for clients to sign on or off, a way to check friends' online
statuses, and a method to request a friend's ip address and port.

\subsection{Server Design Decisions}

We had to make a number of design decisions for the server, which are described here.

We'll start with the choice of language. Because the server is a separate entity from
the chat client, we had some extra freedom for language and design choice. After all,
the only real requirement for the server was that it be able to connect and communicate
with each client. After some consideration, we decided to implement the server in
Python for two main benefits. First, Python has a number of modules, including a 
socket-based asynchronous server, available to aid with development. Second, Python's
general syntax and ease of use makes it a great tool for creating mock-ups or
relatively small, short-term projects. Of course, it works well for larger projects too.
Python made coding much cleaner and more maintainable for the short term goals of our
project.

We wanted the server to perform some of the same capabilities that a watered-down chat
server (such as one used for Skype) might have or use. To that end, we needed it to
have login functionality and tracking. We determined that the server
had to perform the following tasks at minimum:

\begin{enumerate}
  \item Provide a way for a user to register that he or she is online.
  \item Provide a way for users to talk to each other.
  \item Maintain a list of users each client is permitted to contact.
  \item Provide information updates to each user about their friends' online statuses.
\end{enumerate}

The first task is essential because it represents
our intention to add a server element: to keep track of users.
By giving users a way to register that they are connected (and conversely,
disconnected as they become so), the server could keep track of
users as they log in or log out, allowing it to perform additional functionality. This 
is required for task 4 and the basis of our notification system on the client side. 

The second task is important because, in a typical setting, users aren't going to
know the IP address or port number of the machines their friends are on.
The client needs a way of obtaining this information from the server, and thus being
able to chat with others. It is important that this information is maintained by the
server in order to not limit which machines our users can chat on or where they can 
chat from; therefore, a client must request for this information every time he or she
wants to start a chatting session with one of his or her friends.

Strictly speaking, only points 1 and 2 above are actually needed for a minimalistic
server to function. The server keeps track of who is online, and tells clients where
they can contact other clients who are online. However, we wanted our server to
emulate a couple more features seen in commercial media streaming applications. This 
leads to our third task: maintaining a friends list for each client.
The server would keep a list of who is friends with whom, thereby limiting clients
from being able to randomly call or spam other user online.

The final piece of core functionality is closely related to task 3. In order
to cache friend information on the client side (so the client won't repeatedly make requests
for offline users), the server should be capable of periodically sending updates to each
connected client in order to notify them of which of their friends they can call at 
any given time.

Our final core design choice relates to storage of user information. As it is rather 
standard in industry, the server should be capable of storing information in a database 
file that is easily accessible without any parsing or configurations. In this database, the 
server would keep track of user and friend list information, as well as any other necessary
information needed, i.e. ip addresses and port numbers. Although it isn't very much 
information to keep track of, it still gives a relatively simple solution to keeping track 
of users without.
 
\subsection{Server Implementation}
Now that we've covered the key design decisions of the server, let's describe how the server
was implemented. A generalized state diagram is shown in Fig.~\ref{fig:server_diag}.

\begin{figure*}[!t]
   \centering
      \includegraphics[width=0.8\textwidth]{pics/Server_StateDiagram}
   \caption{Server State Diagram demonstrating the general flow of the central server.}
\label{fig:server_diag}
\end{figure*}


Python has a number of different modules available for applications to communicate
through using network sockets. Our first consideration was the SocketServer module. This 
library
provides an interface which can block receive requests, and then execute a request
handler via a callback. Unfortunately, this module has one major flaw: it can only handle
one user request at a time. Because the server needed to handle multiple connected users
asynchronously, we settled on using the asyncore module. This module has essentially
the same functionality as SocketServer, but with asynchronous I/O. In other words, each
new incoming request executes a request handler in its own seperate thread, allowing 
multiple clients to connect and remain logged in at the same time.

In order to allow clients to log on, request information, and receive updates, it made
sense for the server to use a reliable connection-oriented protocol. Thus, TCP was the 
natural
choice. When a client connects to the server, a new thread is spawned to maintain
the connection to the client. If the client makes any requests, the response packets will 
be sent
directly to this child thread instead of the main server loop. Each child thread has its own
instance of a request handler that handles the logic for reading, interpreting, and replying
to each client request. If a client closes its connection, the thread closes as well and is
removed from the loop. 

To log in, a client establishes a connection to the server (in our tests
the client would accept the IP address of the server as an input parameter) and sends it
a packet with two 20 byte strings, a username and a password, padded as necessary with
null bytes. The server then accepts the packet, verifies the username and password, and
replies back to the client with either a success or failure packet depending on the
validity of the credentials. The password is sent over the network as plaintext (normally
it would be encapsulated in an SSL connection) and the server, upon reception, hashes
the password with salt using an SHA-512 function and compares the result to the truth value 
in the database. If the hashes match, the user is granted login priviledges. This
security isn't hack-proof, but it at least models a standard login procedure.  Additional,
security concerns are left for future work.

Once a client successfully logs in, the server replies back with a friend list packet.
This packet contains a list of usernames and ids of all the users a client is currently
friends with. This packet does not contain any online statuses, and in the event of
an overflow (more friends than can fit in the packet) the server would simply truncate
the list. We did not run into any issues with this in testing because we only had 2
users, but in a real-world environment this certainly would be more of a concern and 
therefore is left for future work.

Every few seconds (our implementation configured it to 4), the server sends out an
update to each client currently maintaining a connection to it. To accomplish this,
the server initializes a timer to provide an interrupt which executes a special method
once per interval. This method, when run, determines who has changed status in the past
interval (whether they went online or offline), and sends each of the clients' friends a 
status update. To reduce the overhead of using the same
connection, this status update would be sent via a separate connection that each client
establishes when it first connects. When the method finishes executing, it uses a callback
to set up a new timer interrupt. 

One other capability the server provides to each client is a friend address request. Due to
the nature of our project and the desire to not want to endlessly enter IP addresses and
port information
when performing tests, we implemented the ability for a client to request a friend's IP 
address and port number
information from the server. A client simply sends a packet containing the friend's ID and 
the server, upon verifying the friend is indeed a friend and is online, replies with the
IP address and the port number of that friend's computer. This information is then used
by the client to establish a client-to-client connection that is independent of the 
server.

For sending and receiving binary data over the network, the server uses the Python struct
module, which provides a convenient way to pack and unpack data with specific size
limits. When sending data, the server calculates the values it needs, packs them as
appropriate, and concatenates the values together (the struct module uses strings
for its binary representation, in which an encoded character represents the specific values
we want to send or receive). When receiving data, the server decrypts this encoded character
to identify the type of packet received.

To keep track of users and friends, the server uses a SQLite database consisting of
two tables. The first table, called ``Users,'' has one row per user, with columns
giving the username, user ID, and password hash. The second table, called ``Friends,''
contains a row per user which includes each of the users friends. The design of this is such
that ``A is friends with B'' and ``B is friends with A'' needs to occur in the table
for A and B to properly communicate. In other words, for every set of friends, two rows
in the ``Friends'' table are needed. This is not exactly the most efficient (or even the
most reliable) way to keep track of the information, but it served its purpose well enough
for the scope of this project. When the server wants to obtain information from the
database, it uses the sqlite3 Python module to do so. 



\section{Client Overview}
\label{sec:client_des}

\begin{figure*}[!t]
   \centering
      \includegraphics[width=0.8\textwidth]{pics/Client_StateDiagram}
   \caption{Client State Diagram demonstrating the general flow on the client-side.}
\label{fig:client_state_diag}
\end{figure*}

The client application is the core of the voice chatting system.  It initiates all
the communication with the server in order to receive the user's personal 
information, which includes a friends list. For purposes of this paper, a friend
is a separate client who appears on the list of people a user can call. The application currently
does not support the ability for users to add friends during run-time.  Once the 
user logs into the server,
the client has free roam to make outgoing calls or answer incoming calls.  This
application only supports one-to-one chatting, and has a notification system that 
indicates when a friend is calling while the user is chatting with another client.  

The client handles all the audio transmission between itself and another user
and also manages the setup and teardown of the transport protocol used to
communicate between two clients.  Each client is fitted with a transport protocol
object that handles all the layer three calls which send and receive data between
two clients using sockets.  The specific transport layer protocol is determined
upon loading the application by the user (through a command line parameter). The
three options are DCCP, UDP, and TCP. Each client is capable of calling friends in addition to accepting calls from
other clients.  This requires a \textit{server} and \textit{client} model present
on each client's machine in order to monitor a socket for incoming calls as well
as open a new socket for outgoing calls.  Fig.~\ref{fig:client_state_diag} gives 
a state diagram representation of the client-side system.


The client communicates with the server once upon logging into the system and again
every time a client wants to call a friend.  When making an outgoing call,
the client asks the server for the hostname and port number of the friend he or
she wants to call.  This information is relayed back to the requestor which is then
used to directly call the friend of interest.  Since all communication with the
server contains important information about how to contact others within the 
network, the connection needs to be reliable. As stated earlier, TCP is used for all 
communication between the clients and the server. Lastly, each client starts a seperate thread
immediately after logging into the server that handles receiving
periodic updates (as a single packet) from the server which indicate any changes in a user's friends'
online status. 


\subsection{Transport Layer Abstraction}
\label{subsec:transport_abs}

The main feature of this voice chatting application is its ability to easily 
swap in and out different transport layer protocols.  Essentially, this application
is a framework for testing the behavior of transport layer protocols with an
audio streaming application.  The abstraction was designed using the concept of
polymorphism.  We constructed a class called \textit{TransProtocol} 
with several virtual functions that inheriting children must implement, i.e. DCCP,
TCP, and UDP.  The list of virtual functions includes the following:

\begin{itemize}
   \item{initMaster(uint16\_t)}
   \item{initSlave(char *, uint16\_t)}
   \item{sendPacket(void *, size\_t)}
   \item{recvPacket(void *, size\_t, int flags)}
   \item{getCallerID()}
   \item{ignoreCaller()}
   \item{answerCall()}
   \item{endCall()}
\end{itemize}

The above functions need to be implemented individually according to the specific
transport layer protocol.  For example, DCCP and TCP are connection-oriented and
need to setup a connection between two computers initially before any data packets 
can be transmitted, whereas UDP connections open a socket and wait for 
whomever decides to send data its way.  The initialization steps for each of the
different transport layer protocols occur in the functions \textit{initMaster} and
\textit{initSlave}.  These two functions differ in the way the sockets are handled.

In the \textit{initMaster} function, the socket is bound to a randomly selected 
port, which is then configured to listen for incoming requests. 
Before any clients can call this user, the port number needs to be reported to
the central server. This port is continuously 
monitored by the client to notify the user of any incoming calls.  When an incoming
call is detected, the client machine accepts the call and extracts the caller's 
identification from the first packet to determine who the caller is using the 
\textit{getCallerID} function.  At this point, the user is notified when an 
incoming request has been received and is given the option to accept or decline
the call.  The appropriate action is taken based on the users response.

The \textit{initSlave} function is used when a client wants to make an outgoing
call to a friend.  For this to occur, the client machine requests from the central 
server the hostname and port number of the friend he or she wants to call, who is
identified by the user identification number.  After retrieving the proper contact 
information from the server, \textit{initSlave} opens a socket to 
communicate with the friend.  Similar to the \textit{initMaster} function, if the
callee accepts a user's call, then the chatting state will be entered and audio
capturing and playback will begin. This audio information will be transferred across
the network with the chosen protocol.

To handle the processing of audio transmission and incoming chat requests, the
main application was multi-threaded using the boost threading library. The main
application thread handles the incoming requests and parses through the first
incoming request packets to identify who the caller is. Once a connection
is established between two clients and the callee accepts the call, a chat
thread is spawned that handles the audio capturing and playback along with the 
packet construction and transmission. An audio playback thread is also started, which we discuss in the next section below.


\subsection{Audio Processing}
\label{subsec:audio_proc}

Audio transmission was handled in a fairly simple manner. The ALSA libraries
were used to capture and playback the raw audio from the microphone on each of the 
client's machines. We used them because of their simplicity and
built-in Linux functionality.

The audio was captured at 8,000 bytes per second on 2 channels.  In a 
single packet, 1,400 bytes are transmitted.  The capture and transmission of audio 
data occurs in the main chatting thread while the playback of the received audio 
packets occur in a seperate thread. This thread is dedicated to reading from
a buffer that contains received audio packets. To reduce the latency 
introduced by transmitting in real-time over a network, a minimum buffer size
threshold was set to 5 packets.



\input{Evaluation_2}

\section{Conclusion}
\label{sec:concl}

From the results we can see that ....

\balance
\bibliography{references}
\end{document}
